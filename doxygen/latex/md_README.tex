Bencho is a C++ benchmarking framework that will help you create and run your own benchmarks.

\subsection*{Current Status}

The Bencho Framework can be used already, but it is still in development. Features can change or will be added and even some basic functionality could change in the next versions.

The following features are currently implemented\-:


\begin{DoxyItemize}
\item Create benchmarks by using the Abstract\-Benchmark as template
\item Add Parameters easily
\item Measure different performance counters at the same time
\item Register different test series
\item Supports warm-\/up runs, calibration runs, chache clearing, parameter versions
\item All common aggregating functions included in an Aggregator
\item Save results in C\-S\-V formated files
\item Plot results directly via Gnu\-Plot
\end{DoxyItemize}

\subsection*{Building}

Supported operating systems are Linux and Mac O\-S X, however the P\-A\-P\-I library wont work on the Apple operating systems.

To build the Bencho framework you will only need to execute \begin{DoxyVerb}make
\end{DoxyVerb}


If not specified already, it will automatically ask you for your prefered Configuration of the options production mode, papi use and verbose build. If you want to change these settings later just execute \begin{DoxyVerb}make config
\end{DoxyVerb}


The Bencho framework is now ready to use. On how to include this framework in your project best, please take a look at the \href{https://github.com/schwald/benchosample}{\tt Bencho Sample} or at the documentation.

\subsection*{Documentation}

The Bencho project provides a detailed documentation on how to use and set up the framework. The ready build documentation can be found in the docs directory.

\subsection*{License}

Bencho is licensed as open source after the Open\-Source \char`\"{}\-Licence of the Hasso-\/\-Plattner Institute\char`\"{} declared in the L\-I\-C\-E\-N\-S\-E file of this project.

Question\-: Why does Bencho not use M\-I\-T or B\-S\-D or X\-X license. The reason for our approach comes with the German copyright law. Common B\-S\-D and M\-I\-T licenses are not necessarily compatible and thus can potentially lead to problems. To overcome this problem this project uses an specifically designed open source license to be compatible with German law. The most prominent difference is the exclusion of all liabilities which is not possible in Germany. 